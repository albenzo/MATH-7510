\documentclass[10pt]{article}
\usepackage[utf8]{inputenc}
\usepackage{amscd}
\usepackage{amsmath}
\usepackage{amssymb}
\usepackage{amsthm}
\usepackage{listings}
\usepackage{enumerate}

\textwidth=15cm \textheight=22cm \topmargin=0.5cm \oddsidemargin=0.5cm \evensidemargin=0.5cm

\newcommand{\sk}{\vskip 10mm}
\newcommand{\bb}[1]{\mathbb{#1}}
\newcommand{\ra}{\rightarrow}

\theoremstyle{plain}
\newtheorem{problem}{Problem}
\newtheorem{lemma}{Lemma}[problem]

\theoremstyle{remark}
\newtheorem{tpart}{}[problem]
\newtheorem*{ppart}{}

\begin{document}

\begin{problem} %1
  
\end{problem}

\begin{proof}
  
\end{proof}

\sk

\begin{problem} %2
  
\end{problem}

\begin{proof}
  Let $p:E\rightarrow B$ be a covering map where both $B$ and $E$ are path connected.
  Now suppose that the cover is normal. Then consider a loop $\alpha$ in $B$.
  Since $E$ is a normal covering space for any $e\in p^{-1}(b_0)$ there
  is a homeomorphism such that $h(e_0)=e$ and that $p\circ h=p$. Then
  if we consider a lift $\alpha$ to $\widetilde{\alpha}$ in $E$ it will either
  lift to a loop or a path uniquely. If we apply $h$ to $\widetilde{\alpha}$
  then we have $h\circ\widetilde{\alpha}$ as a loop or path based at $e$ however
  since lifts are unique this is precisely the lift of $\alpha$ to the
  covering space when the basepoint is shifted. Since the cover is
  normal we can do this for all basepoints. Therefore all lifts
  of $\alpha$ are either loops or paths.

  Otherwise suppose that given a loop in $B$ either all of its lifts
  are loops or none of them are loops. Then consider $[\alpha]\in\pi_1(B,b_0)$ and
  $[\gamma]\in p_*(\pi_1(E,e_0))$. Then consider the lift of $\alpha*\gamma*\alpha^{-1}$. If
  $\alpha$ lifts to a loop then $\alpha*\gamma*\alpha^{-1}$ will lift to a loop which
  implies that $[\alpha*\gamma*\alpha^{-1}]\in p_*(\pi_1(E,e_0))$. Otherwise
  if $\alpha^{-1}$ lifts to a path then so does $\alpha$. The path
  $\widetilde{\alpha}$ will go to some other point $e\in p^{-1}(b_0)$. Since
  $\gamma$ lifts to a loop at $e_0$ it will also lift to a loop based at $e$ which
  after concatenating the lift of $\alpha^{-1}$ we get back to $e_0$ which shows
  that $\alpha*\gamma*\alpha^{-1}$ lifts to a loop and therefore $[\alpha*\gamma*\alpha^{-1}]\in p_*(\pi_1(E,e_0))$.

  Therefore a covering is normal if and only if for each loop in $B$,
  either all its lifts are loops or none of them are loops.
\end{proof}

\sk

\begin{problem} %3
  
\end{problem}

\begin{proof}
  
\end{proof}

\sk

\begin{problem} %4
  
\end{problem}

\begin{proof}
  Let $\Sigma_n$ and $\Sigma_m$ be closed surfaces of genus $n$ and $m$
  respectively. Suppose that $m=n$. Then we can express
  both surfaces as a polygons with $4n$ sides glued together.
  However this would mean that $\chi(\Sigma_n)=\chi(\Sigma_m)$ which by
  Problem $1$ means that they are a 1 sheeted covering
  spaces of each other and as such $\Sigma_n\cong \Sigma_m$.

  Otherwise suppose that $\Sigma_m\cong\Sigma_n$. Then any cell complex
  for $\Sigma_m$ would also be one for $\Sigma_n$ and vice versa.
  Since $\Sigma_m$ is of genus $m$ it has a cell complex
  with $4m$ sides. Since this is also a cell complex
  for $\Sigma_n$ we know that $n\leq m$. However by the same
  reasoning we can show that $m\leq n$ implying that
  $n=m$.

  Therefore $\Sigma_n\cong\Sigma_m$ if and only if $m=n$.
\end{proof}

\sk

\begin{problem} %5
  
\end{problem}

\begin{proof}
  Let $p:E\rightarrow B$ be a covering map with $B$ and $E$ both path connected. Now
  suppose that the group of Deck transformations acts transitively
  on the set of points which lie over any single point in $B$. Then
  it acts transitively on $p^{-1}(b_0)$ which implies that the
  cover is normal.

  Otherwise suppose that $E$ is a normal cover. Then the group of Deck
  transformations acts transitively on the basepoint. As
  $B$ is path connected for any point $b\in B$ there is
  an isomorphism  $\varphi:\pi_1(B,b_0)\rightarrow \pi_1(B,b)$ made by going along
  a path $\gamma$ from $b$ to $b_0$ and back. This path has a lift
  $\widetilde{\gamma}$ which gives rise to an isomorphism from
  $\pi_1(E,e_0)$ to $\pi_1(E,e)$ where $e\in p^{-1}(b)$. Then we
  have $(E,e)$ as a covering space for $(B,b)$ with the
  same map and then $p_*(\pi_1(E,e))$ will be normal in
  $\pi_1(B,b)$. Thus the group of Deck transformations will
  act transitively on $p^{-1}(b)$ and since this was for
  an arbitrary point it holds for all points $b\in B$.

  Therefore a cover is normal if and only if the group of
  Deck transformations acts transitively on the set of points
  in $E$ which lie over any single point in $B$.
\end{proof}

\sk

\begin{problem} %6
  
\end{problem}

Using the Seifert van Kampen theorem we can compute that the
fundamental group of $\bb{R}P^2\vee \bb{R}P^2$ is
\[ \pi_1(\bb{R}P^2\vee \bb{R}P^2,*)\cong \langle a,b|a^2,b^2\rangle\]
The subgroups for this group are:
\begin{itemize}
\item $\langle e\rangle$
\item $\langle a\rangle$
\item $\langle b\rangle$
\item $\langle (ab)^n\rangle$ for all $n\in \bb{N}$
\item $\langle a,b|a^2,b^2\rangle$
\end{itemize}

The covering spaces, with pictures below, that correspond to having
the above group of Deck transformations are:

\begin{itemize}
\item $\bb{R}P^2\vee \bb{R}P^2$
\item $S^2$ with two copies of $\bb{R}P^2$ wedged on at antipodal points where
  crossing the sphere corresponds to a loop in $a$.
\item As the previous but with $b$ as the crossing of the sphere.
\item A wreath of an even number of spheres wedged together.
\item An infinite chain of spheres wedged together at antipodal points.
\end{itemize}


\sk

\begin{problem} %7
  
\end{problem}

\begin{proof}
  Let $X$ be a path connected, locally path connected, semi-locally simply
  connected space. Then let $H:=[\pi_1(X,*),\pi_1(X,*)]$ where
  $[G,G]$ denoted the commutator subgroup and let $\widetilde{X}_{ab}$
  be the covering space associated with $H$ such that
  $p_*(\pi_1(\widetilde{X}_{ab},\widetilde{*}))=H$. Then since $H$ is normal
  $\widetilde{X}_{ab}$ is a normal covering space and its Deck group is
  $\text{Deck}(\widetilde{X}_{ab})=N(H)/H=\pi_1(X,*)/H$
  which is Abelian. Moreover any quotient of $\pi_1(X,*)$ with a normal
  subgroup $\pi_1(X,*)/N$ that is abelian must have $N$ containing $H$ as a
  subgroup. \textbf{Add why this means $\widetilde{X}_{ab}$ covers them}.

  The universal abelian cover for $S^1\vee S^1$ is the 2d grid of real lines
  with a copy of $\bb{R}$ for each integer. As shown below.
  \textbf{Actually insert the picture}.

  The preimage of the basepoint are each of the intersections. The
  Deck group for the space is $\bb{Z}\times\bb{Z}$.

  For $S^1\vee S^1 \vee S^1$ the universal abelian cover is similar however
  it is a 3d grid of real lines with a copy of $\bb{R}$ for each integer.
  \textbf{Insert a picture}

  The preimage of the basepoint will be the intersections. The Deck group
  of the space is $\bb{Z}\times\bb{Z}\times\bb{Z}$.
\end{proof}

%%%%%%%%%%%%%%%%%%%%%%%%%%%%%%%%%%%%%%%%%%%%%%%%%%%%%%%%%%%%%%%%%%%%%%%%%%%%%
\end{document}
