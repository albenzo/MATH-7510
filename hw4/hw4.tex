\documentclass[10pt]{article}
\usepackage[utf8]{inputenc}
\usepackage{amscd}
\usepackage{amsmath}
\usepackage{amssymb}
\usepackage{amsthm}
\usepackage{listings}
\usepackage{enumerate}

\textwidth=15cm \textheight=22cm \topmargin=0.5cm \oddsidemargin=0.5cm \evensidemargin=0.5cm

\newcommand{\sk}{\vskip 10mm}
\newcommand{\bb}[1]{\mathbb{#1}}
\newcommand{\ra}{\rightarrow}
\newcommand{\wt}[1]{\widetilde{#1}}

\theoremstyle{plain}
\newtheorem{problem}{Problem}
\newtheorem{lemma}{Lemma}[problem]

\theoremstyle{remark}
\newtheorem{tpart}{}[problem]
\newtheorem*{ppart}{}

\begin{document}

\begin{problem} %1
  
\end{problem}

\begin{proof}
  Let $\Sigma,\Sigma'$ be closed orientable surfaces such that $\chi(\Sigma)=n\chi(\Sigma')$ with $n\in\bb{Z}_{>0}$.
  Since $\Sigma,\Sigma'$ are closed and orientable we can write their Euler characteristic
  in terms of the genus $(2-2g=\chi)$ of the spaces where in this case the genus will be
  equivalent to the number holes in the space. This gives us
  \[ 2-2g = n(2-2g') \Rightarrow 1+n(g'-1)=g\]
  These numbers correspond in such a way that we can construct the covering
  space as a single hole in the space with lines of holes length $g'-1$ emerging.
  This will give us $n$ copies of $\Sigma'$ attached at the center from a continuous
  deformation of $\Sigma$. Therefore $\Sigma$ is an $n-$fold covering space of $\Sigma'$.
\end{proof}

In the case where the surfaces are nonorientable both surfaces will be made of
handles or crosscaps attached to a sphere. In the scenario where it is only crosscaps
the Euler characteristic in terms of the genus is $\chi(\Sigma)=2-g$ in which case
the analogous equation to the above would be $2+n(g'-2)=g$. So we could do a
similar construction with a line of two crosscaps in the center which would
combine to be a torus. However
if there is a mix of crosscaps and handles we would need to know that the
numbers in both surfaces correspond to ensure that we have a covering.

\sk

\begin{problem} %2
  
\end{problem}

\begin{proof}
  Let $p:E\rightarrow B$ be a covering map where both $B$ and $E$ are path connected.
  Now suppose that the cover is normal. Then consider a loop $\alpha$ in $B$.
  Since $E$ is a normal covering space for any $e\in p^{-1}(b_0)$ there
  is a homeomorphism such that $h(e_0)=e$ and that $p\circ h=p$. Then
  if we consider a lift $\alpha$ to $\widetilde{\alpha}$ in $E$ it will either
  lift to a loop or a path uniquely. If we apply $h$ to $\widetilde{\alpha}$
  then we have $h\circ\widetilde{\alpha}$ as a loop or path based at $e$ however
  since lifts are unique this is precisely the lift of $\alpha$ to the
  covering space when the basepoint is shifted. Since the cover is
  normal we can do this for all basepoints. Therefore all lifts
  of $\alpha$ are either loops or paths.

  Otherwise suppose that given a loop in $B$ either all of its lifts
  are loops or none of them are loops. Then consider $[\alpha]\in\pi_1(B,b_0)$ and
  $[\gamma]\in p_*(\pi_1(E,e_0))$. Then consider the lift of $\alpha*\gamma*\alpha^{-1}$. If
  $\alpha$ lifts to a loop then $\alpha*\gamma*\alpha^{-1}$ will lift to a loop which
  implies that $[\alpha*\gamma*\alpha^{-1}]\in p_*(\pi_1(E,e_0))$. Otherwise
  if $\alpha^{-1}$ lifts to a path then so does $\alpha$. The path
  $\widetilde{\alpha}$ will go to some other point $e\in p^{-1}(b_0)$. Since
  $\gamma$ lifts to a loop at $e_0$ it will also lift to a loop based at $e$ which
  after concatenating the lift of $\alpha^{-1}$ we get back to $e_0$ which shows
  that $\alpha*\gamma*\alpha^{-1}$ lifts to a loop and therefore $[\alpha*\gamma*\alpha^{-1}]\in p_*(\pi_1(E,e_0))$.

  Therefore a covering is normal if and only if for each loop in $B$,
  either all its lifts are loops or none of them are loops.
\end{proof}

\sk

\begin{problem} %3
  
\end{problem}

\begin{proof}\ \\
  \begin{itemize}
  \item[(a)] Let $B$ be a path connected, locally path connected,
    semi-locally simply connected space and let $p:E\rightarrow B$ be an $n-$sheeted
    covering space. Define a map $P_\gamma:p^{-1}(b_0)\rightarrow p^{-1}(b_0)$ for
    $[\gamma]\in\pi_1(B,b_0)$by $e_i\mapsto e_j$
    if the lift $\wt{\gamma}$ based at $e_i$ fulfills $\gamma(0)=e_i$ and $\gamma(1)=e_j$.
    The map $P_\gamma$ is a bijection since $P_{\gamma^{-1}}$ will send $e_j$ to $e_i$
    for any $e_i$ and as such $P_\gamma$ has an inverse. Therefore we can
    define a map $\varphi_E:\pi_1(B,b_0)\rightarrow S_n$ by $\gamma\mapsto P_\gamma$.
    
    To show that $\varphi_E$ is a homomorphism consider $[\alpha],[\beta]\in\pi_1(B,b_0)$.
    Suppose that $P_\alpha(e_i)=e_j$ and $P_\beta(e_j)=e_k$. Then for some $e_i$
    \[ \varphi_E(\beta)\circ\varphi_E(\alpha)(e_i)=P_\beta\circ P_\alpha(e_i)=e_k=P_{\alpha*\beta}(e_i)=\varphi_E(\alpha*\beta)\]
    Therefore $\varphi_E:\pi_1(B,b_0)\rightarrow S_n$ is a homomorphism. This representation
    of $\pi_1(B,b_0)$ is dependent on the covering space. Therefore we can
    construct a map $\rho$ from the set of equivalence classes of $n-$sheeted
    covering spaces and representations of $\pi_1(B,b_0)$ via
    $E\mapsto \varphi_E$.

    We will show that $\rho$ is a bijection by constructing its inverse.
    Let $\psi:\pi_1(B,b_0)\rightarrow S_n$ be a representation of $\pi_1(B,b_0)$. First
    let $\wt{E}$ be the universal cover of $B$ viewed as homotopy
    classes of paths based at $b_0$ and consider the space
    $\wt{E}\times \{0,\ldots,n\}$. We can define the covering space
    $E_\psi$ as a quotient space of $\wt{E}$. First define a
    map $h:\wt{E}\times\{0,\ldots,n\}$ as $h([\gamma],e_i)=\wt{\gamma}(1)$ where
    $\wt{\gamma}$ is the lift of $\gamma$ based at $e_i$. Consider points
    $e_i,e_j,$ and $e_k$ such that $P_\alpha(e_i)=e_k$. Then if
    there is a path $\gamma$ such that $h([\gamma],e_i)=e_k$
    and $h([\gamma],e_j)=e_k$ then $e_i\sim e_j$. This will collapse
    the universal cover such that the action $\psi$ is well defined
    on $X/\sim$ Therefore the quotient is a covering space that
    realizes the action associated with the action associated
    with $\psi$ up to equivalence of covering spaces.

    Therefore the equivalences classes of $n-$sheeted covering
    spaces are in bijection with representations of $\pi_1(B,b_0)$
    in $S_n$
  \item[(b)] Suppose that $\wt{X}$ is path connected. Then
    consider the path from $\alpha$ in $\wt{X}$ such that
    $\alpha(0)=e_i$ and $\alpha(1)=e_j$. Then $p\circ\alpha$ is a loop that
    such that the action will permute $e_i$ and $e_j$.
    Since we can do this for any such $i$ and $j$
    the representation is transitive.

    Otherwise suppose that the representation
    $\varphi:\pi_1(B,b_0)\rightarrow S_n$ is transitive. Then given
    $e_i\in p^{-1}(b_0)$ there is a path to $e_j\in p^{-1}(b_0)$.
    The map from the fundamental group based at $b_0$ to
    some $b\in B$ induces an isomorphism of fundamental groups
    so the action will remain transitive. As such by the
    same reasoning we can always go between preimages of
    a point. However since $B$ is path connected if
    we lift a path from $b$ to $b'$ this will lift
    to a path between two specific preimages giving us
    a way to construct paths between any two points of $E$.

    Therefore $\wt{X}$ is path connected if and only if
    the representation is transitive.
  \end{itemize}
\end{proof}

\sk

\begin{problem} %4
  
\end{problem}

\begin{proof}
  Let $\Sigma_n$ and $\Sigma_m$ be closed surfaces of genus $n$ and $m$
  respectively. Suppose that $m=n$. Then we can express
  both surfaces as a polygons with $4n$ sides glued together.
  However this would mean that $\chi(\Sigma_n)=\chi(\Sigma_m)$ which by
  Problem $1$ means that they are a 1 sheeted covering
  spaces of each other and as such $\Sigma_n\cong \Sigma_m$.

  Otherwise suppose that $\Sigma_m\cong\Sigma_n$. Then any cell complex
  for $\Sigma_m$ would also be one for $\Sigma_n$ and vice versa.
  Since $\Sigma_m$ is of genus $m$ it has a cell complex
  with $4m$ sides. Since this is also a cell complex
  for $\Sigma_n$ we know that $n\leq m$. However by the same
  reasoning we can show that $m\leq n$ implying that
  $n=m$.

  Therefore $\Sigma_n\cong\Sigma_m$ if and only if $m=n$.
\end{proof}

\sk

\begin{problem} %5
  
\end{problem}

\begin{proof}
  Let $p:E\rightarrow B$ be a covering map with $B$ and $E$ both path connected. Now
  suppose that the group of Deck transformations acts transitively
  on the set of points which lie over any single point in $B$. Then
  it acts transitively on $p^{-1}(b_0)$ which implies that the
  cover is normal.

  Otherwise suppose that $E$ is a normal cover. Then the group of Deck
  transformations acts transitively on the basepoint. As
  $B$ is path connected for any point $b\in B$ there is
  an isomorphism  $\varphi:\pi_1(B,b_0)\rightarrow \pi_1(B,b)$ made by going along
  a path $\gamma$ from $b$ to $b_0$ and back. This path has a lift
  $\widetilde{\gamma}$ which gives rise to an isomorphism from
  $\pi_1(E,e_0)$ to $\pi_1(E,e)$ where $e\in p^{-1}(b)$. Then we
  have $(E,e)$ as a covering space for $(B,b)$ with the
  same map and then $p_*(\pi_1(E,e))$ will be normal in
  $\pi_1(B,b)$. Thus the group of Deck transformations will
  act transitively on $p^{-1}(b)$ and since this was for
  an arbitrary point it holds for all points $b\in B$.

  Therefore a cover is normal if and only if the group of
  Deck transformations acts transitively on the set of points
  in $E$ which lie over any single point in $B$.
\end{proof}

\sk

\begin{problem} %6
  
\end{problem}

Using the Seifert van Kampen theorem we can compute that the
fundamental group of $\bb{R}P^2\vee \bb{R}P^2$ is
\[ \pi_1(\bb{R}P^2\vee \bb{R}P^2,*)\cong \langle a,b|a^2,b^2\rangle\]
The subgroups for this group are:
\begin{itemize}
\item $\langle e\rangle$
\item $\langle a\rangle$
\item $\langle b\rangle$
\item $\langle (ab)^n\rangle$ for all $n\in \bb{N}$
\item $\langle a,b|a^2,b^2\rangle$
\end{itemize}

The covering spaces, with pictures attached, that correspond to having
the above group of Deck transformations are:

\begin{itemize}
\item $\bb{R}P^2\vee \bb{R}P^2$ which projects to $\pi_1(\bb{R}P^2\vee \bb{R}P^2,*)$
\item $\bb{R}P^2$ where $b$ lifts to a constant loop. This corresponds to
  the subgroup $\langle a\rangle$
\item $\bb{R}P^2$ where $a$ lifts to a constant loop. This corresponds to
  the subgroup $\langle b\rangle$
\item A wreath of $2n$ spheres wedged together at distinct points.
  This projects to the subgroup $\langle (ab)^n\rangle$
\item An infinite chain of spheres wedged together at antipodal points.
  This projects to the subgroup $\{e\}$.
\end{itemize}

Since we have a covering space for each subgroup of $\langle a,b|a^2,b^2\rangle$ and
covering spaces that project to the same subgroup are equivalent we
can conclude that these characterize all covering spaces of $\bb{R}P^2\vee \bb{R}P^2$.

\sk

\begin{problem} %7
  
\end{problem}

\begin{proof}
  Let $X$ be a path connected, locally path connected, semi-locally simply
  connected space. Then let $H:=[\pi_1(X,*),\pi_1(X,*)]$ where
  $[G,G]$ denoted the commutator subgroup and let $\widetilde{X}_{ab}$
  be the covering space associated with $H$ such that
  $p_*(\pi_1(\widetilde{X}_{ab},\widetilde{*}))=H$. Then since $H$ is normal
  $\widetilde{X}_{ab}$ is a normal covering space and its Deck group is
  $\text{Deck}(\widetilde{X}_{ab})=N(H)/H=\pi_1(X,*)/H$
  which is Abelian. Moreover any quotient of $\pi_1(X,*)$ with a normal
  subgroup $\pi_1(X,*)/N$ that is abelian must have $N$ containing $H$ as a
  subgroup. Thus $\wt{X}_{ab}$ must cover the covering space corresponding
  to $N$.

  The universal abelian cover for $S^1\vee S^1$ is the 2d grid of real lines
  with a copy of $\bb{R}$ for each integer.

  The preimage of the basepoint are each of the intersections. The
  Deck group for the space is $\bb{Z}\times\bb{Z}$.

  For $S^1\vee S^1 \vee S^1$ the universal abelian cover is similar however
  it is a 3d grid of real lines with a copy of $\bb{R}$ for each integer.
  Both pictures are attached.

  The preimage of the basepoint will be the intersections. The Deck group
  of the space is $\bb{Z}\times\bb{Z}\times\bb{Z}$.
\end{proof}

%%%%%%%%%%%%%%%%%%%%%%%%%%%%%%%%%%%%%%%%%%%%%%%%%%%%%%%%%%%%%%%%%%%%%%%%%%%%%
\end{document}
