\documentclass[10pt]{article}
\usepackage[utf8]{inputenc}
\usepackage{amscd}
\usepackage{amsmath}
\usepackage{amssymb}
\usepackage{amsthm}
\usepackage{listings}
\usepackage{enumerate}

\textwidth=15cm \textheight=22cm \topmargin=0.5cm \oddsidemargin=0.5cm \evensidemargin=0.5cm

\newcommand{\sk}{\vskip 10mm}
\newcommand{\bb}[1]{\mathbb{#1}}
\newcommand{\ra}{\rightarrow}

\theoremstyle{plain}
\newtheorem{problem}{Problem}
\newtheorem{lemma}{Lemma}[problem]

\theoremstyle{remark}
\newtheorem{tpart}{}[problem]
\newtheorem*{ppart}{}

\begin{document}

\begin{problem}
  
\end{problem} %1

\begin{proof}
  Let $X$ be a space such that all paths having the same endpoints are fixed-endpoint
  homotopic. Then given a loop $\alpha$ based as some point $x_0$ this loop is homotopic
  to the constant map at $x_0$ which implies that $\pi_1(X,x_0)$ is trivial for all
  $X$. Therefore $X$ is simply connected.

  Now suppose that $X$ is a simply connected space. Let $\alpha,\beta$ be loops from $x_0$ to
  $x_1$. For a fixed $t'\in [0,1]$ using path connectedness we can construct a loop
  $l_{t'}$ such that $l_{t'}(0)=l_{t'}(1)=\alpha(t')$, $l_{t'}(\frac{1}{2})=\beta(t)$ and
  that the function $p(t)=l_{t}(\frac{1}{2})$ is a path \textbf{Why?}.
  Since $X$ is simply connected there is a homotopy $L_{t'}$ from $l_{t'}$ to the
  constant map at $\alpha(t')$. Then we can construct a homotopy $H$ from $\beta$ to $\alpha$ as
  \[ H(t,s)= L_t(\frac{1}{2},s) \]
  which will be continuous because of the continuity of the $L_{t'}$s and that
  we choose $l_{t'}$ such that as we vary over $t$ there is a path.

  Therefore a space $X$ is simply connected if and only if all paths with the same
  endpoints are fixed endpoint homotopic.
\end{proof}

\sk

\begin{problem} %2
  
\end{problem}

\begin{proof}
  Let $f:(X,x_0)\rightarrow (Y,y_0)$ and $g:(Y,y_0)\rightarrow (Z,z_0)$. For a continuous map
  $h$ we have $h_*$ defined as $h_*([\gamma])=[h\circ \gamma]$. Then if we consider
  $(g\circ f)_*$:
  \[ (g\circ f)_*([\gamma]) = [(g\circ f)\circ \gamma] = [g\circ (f\circ \gamma)] = g_*([f\circ\gamma])=g_*\circ f_*([\gamma])\]

  Therefore $(g\circ f)_*=g_*\circ f_*$.
\end{proof}

\sk

\begin{problem} %3
  
\end{problem}

\begin{proof}
  
\end{proof}

\sk

\begin{problem} %4
  
\end{problem}

\begin{proof}
  
\end{proof}

\sk

\begin{problem} %5
  
\end{problem}

\begin{proof}
  
\end{proof}

\sk

\begin{problem} %6
  
\end{problem}

\begin{proof}
  
\end{proof}
  
%%%%%%%%%%%%%%%%%%%%%%%%%%%%%%%%%%%%%%%%%%%%%%%%%%%%%%%%%%%%%%%%%%%%%%%%%%%%%
\end{document}

