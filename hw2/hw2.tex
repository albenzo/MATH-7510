\documentclass[10pt]{article}
\usepackage[utf8]{inputenc}
\usepackage{amscd}
\usepackage{amsmath}
\usepackage{amssymb}
\usepackage{amsthm}
\usepackage{listings}
\usepackage{enumerate}

\textwidth=15cm \textheight=22cm \topmargin=0.5cm \oddsidemargin=0.5cm \evensidemargin=0.5cm

\newcommand{\sk}{\vskip 10mm}
\newcommand{\bb}[1]{\mathbb{#1}}
\newcommand{\ra}{\rightarrow}

\theoremstyle{plain}
\newtheorem{problem}{Problem}
\newtheorem{lemma}{Lemma}[problem]

\theoremstyle{remark}
\newtheorem{tpart}{}[problem]
\newtheorem*{ppart}{}

\begin{document}

\begin{problem}
  
\end{problem} %1

\begin{proof}
  Let $X$ be a space such that all paths having the same endpoints are fixed-endpoint
  homotopic. Then given a loop $\alpha$ based as some point $x_0$ this loop is homotopic
  to the constant map at $x_0$ which implies that $\pi_1(X,x_0)$ is trivial for all
  $X$. Therefore $X$ is simply connected.

  Now suppose that $X$ is simply connected and let $\alpha,\beta$ be paths from the point $x_0$
  to $x_1$. Consider the path $\alpha*\beta^{-1}*\beta$. Since $\alpha*\beta^{-1}$ is a loop based at $x_0$
  it is homotopic to the constant map at $x_0$. This implies that
  $\alpha*\beta^{-1}*\beta$ is homotopic to $c_{x_0}*\beta\sim \beta$. Now if we look at $\beta^{-1}*\beta$ this is
  homotopic to the constant path based at $x_1$ and by the same reasoning
  $\alpha*\beta^{-1}*\beta\sim \alpha$. By transitivity this implies that $\alpha\sim \beta$.

  Therefore a space $X$ is simply connected if and only if all paths are fixed
  endpoint homotopic.
\end{proof}

\sk

\begin{problem} %2
  
\end{problem}

\begin{proof}
  Let $f:(X,x_0)\rightarrow (Y,y_0)$ and $g:(Y,y_0)\rightarrow (Z,z_0)$. For a continuous map
  $h$ we have $h_*$ defined as $h_*([\gamma])=[h\circ \gamma]$. Then if we consider
  $(g\circ f)_*$:
  \[ (g\circ f)_*([\gamma]) = [(g\circ f)\circ \gamma] = [g\circ (f\circ \gamma)] = g_*([f\circ \gamma])=g_*\circ f_*([\gamma])\]

  Therefore $(g\circ f)_*=g_*\circ f_*$.
\end{proof}

\sk

\begin{problem} %3
  
\end{problem}

\begin{proof}
  Let $p:E\rightarrow B$ be a covering map with $p(e_0)=b_0$. Let $F:[0,1]^2\rightarrow B$ be continuous
  with $F(0,0)=b_0$. For each point $b\in B$ let $U_b$ be an open neighborhood of $b$ such
  that the preimage under $p$ is the union of disjoint open sets homeomorphic to $U_b$. 
  Then $\{F^{-1}(U_b)\}$ is an open cover of $[0,1]^2$. Let $\{F^{-1}(U_\beta)\}$ be a finite subcover
  of $\{F^{-1}(U_b)\}$. Choose a $U_\beta$ that contains $b_0$. Then for
  $p^{-1}(U_\beta)=\coprod V_\alpha$ choose the $V_{\alpha'}$ containing $e_0$ and define
  $\tilde{F}:F^{-1}(U_\beta)\rightarrow E$ as equivalent to $F|_{F^{-1}(U_\beta)}$ as $V_{\alpha'}$ is homeomorphic
  to $U_\beta$.

  Then for each $\gamma$ such that $F^{-1}(U_\gamma)\cap F^{-1}(U_\beta)\neq \phi$  pick the $V_{\gamma'}$ from
  $p^{-1}(U_\gamma)=\coprod_\gamma V_\gamma$ such that $V_{\gamma'}\cap V_{\alpha'}\neq \phi$. Define $\tilde{F}:F^{-1}(U_\gamma)\rightarrow E$ in
  the same way as before such that the $\tilde{F}$s agree on the common components of their
  image. Then repeat this process for all $U_\beta$ in our finite subcover.
  Since we have a finite number of $\tilde{F}$s and they are defined to be equal on
  the common pieces on their image we can create $\tilde{F}:[0,1]^2\rightarrow E$ using the pasting
  lemma such that $p\circ \tilde{F}=F$. As there was only one choice to make at any given
  point of the construction this lift is unique.

  Now suppose that $F$ is a fixed endpoint homotopy and consider $\tilde{F}$ defined
  as above. Then $F|_{[0]\times[0,1]}$ and $F|_{[1]\times[0,1]}$ are both constant at $b_0$.
  This implies that in order for $p\circ \tilde{F}$ to equal $F$ both
  $\tilde{F}|_{[0]\times[0,1]}$ and $\tilde{F}|_{[1]\times[0,1]}$ must remain in $p^{-1}(b_0)$.
  However by the definition of a covering map each element of $p^{-1}(b_0)$ lies in
  a disjoint open set. Therefore $\tilde{F}$ must also be a fixed endpoint homotopy.
  
  
  Therefore if $p:E\rightarrow B$ is a covering map and $F:[0,1]^2\rightarrow B$ is a homotopy then
  there exists a lift $\tilde{F}:[0,1]^2\rightarrow E$ where if $F$ is fixed endpoint homotopic
  then so is $\tilde{F}$.
\end{proof}

\sk

\begin{problem} %4
  
\end{problem}

\begin{proof}
  Let $E$ be a pointed space, $B$ a connected pointed space, $p:E\rightarrow B$ a covering map,
  and let $|p^{-1}(b)|=n$.
  Define $X_n:=\{b\in B||p^{-1}(b)|=n\}$. Let $x$ be a point in $X_n$ and let $U_x$ be
  an open neighborhood of $x$. Then the preimage $p^{-1}(U_x)$ is $\coprod_{k=1}^nV_k$
  which implies that all other points in $U_x$ are in $X_n$. It then follows that
  $U_x\subset X_n$. Since we did this for an arbitrary point of $X_n$ this implies that
  all points of $X_n$ are interior and as such $X_n$ is open. Then we can express
  $X$ as
  \[ X=\bigcup_nX_n\] where $X_i\cap X_j=\phi$ if $i\neq j$. If there were two or more values for
  which $X_n$ was nonempty this would form a partition of $B$ contradicting the assumption
  that the space is connected.

  Therefore if $p:E\rightarrow B$ is a covering map, $B$ is connected and if there exists
  a $b\in B$ such that $|p^{-1}(b)|=n$ then $|p^{-1}(b')|=n$ for all $b'\in B$.
\end{proof}

\sk

\begin{problem} %5
  
\end{problem}

\begin{lemma}
  If $p:(E,e_0)\rightarrow (B,b_0)$ is a covering map, then if $E$ is path connected
  there exists a surjective map $\phi:\pi_1(B,b_0)\rightarrow p^{-1}(b_0)$.
\end{lemma}

\begin{proof}
  Define $\phi([\gamma])$ as the endpoint other than $e_0$ for the lift $\tilde{\gamma}$.
  To show that this is surjective let $\alpha$ be a path from $e_0$ to $f\in p^{-1}(b_0)$.
  Then $p\circ\alpha$ is a loop in $B$ for whom $\phi([p\circ\alpha])=f$.
\end{proof}

\begin{proof}
  Let $B$ be simply connected, $E$ path connected, and $p:E\rightarrow B$ a covering map.
  By the previous lemma since $E$ is path connected it follows that there exists a
  surjective map \\$\phi:\pi_1(B,b_0)\rightarrow p^{-1}(b_0)$. However since $B$ is simply connected
  the fundamental group is trivial which implies that $p^{-1}(b_0)$ has only one
  element. Since $B$ is simply connected it is also connected so it follows that
  $p^{-1}(b)$ contains only a single element for all $b\in B$. As there is only one
  copy of $B$ in $E$ this implies that $p|_E=p$ and as such $p$ is a homeomorphism
  by the definition of a covering map..

  Therefore if $B$ is simply connected, then any covering map for which $E$ is
  path connected is a homeomorphism.
\end{proof}

\sk

\begin{problem} %6
  
\end{problem}

\begin{proof}
  Let $h:(X,x_0)\rightarrow (Y,y_0)$ be an inessential map and let $[\gamma]\in \pi(X,x_0)$.
  Since $h$ is inessential it follows that there is a homotopy $H$ that
  sends $h$ to the constant map at $y_0$. If we take $h_*([\gamma])=[h\circ\gamma]$
  then we can create a homotopy $H'(t,s)=H(\gamma(t),s)$ that will take $\gamma$
  to the constant map which implies that $h_*([\gamma])=[y_0]$. Since
  this happens for an arbitrary loop in $\pi_1(X,x_0)$ it follows that
  the map $h_*$ is trivial.

  Therefore if $h:(X,x_0)\rightarrow (Y,y_0)$ is inessential then the induced homomorphism
  $h_*$ is trivial.
\end{proof}
  
%%%%%%%%%%%%%%%%%%%%%%%%%%%%%%%%%%%%%%%%%%%%%%%%%%%%%%%%%%%%%%%%%%%%%%%%%%%%%
\end{document}

