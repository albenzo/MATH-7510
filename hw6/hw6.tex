\documentclass[10pt]{article}
\usepackage[utf8]{inputenc}
\usepackage{amscd}
\usepackage{amsmath}
\usepackage{amssymb}
\usepackage{amsthm}
\usepackage{listings}
\usepackage{enumerate}
\usepackage[all,cmtip]{xy}

\textwidth=15cm \textheight=22cm \topmargin=0.5cm \oddsidemargin=0.5cm \evensidemargin=0.5cm

\newcommand{\sk}{\vskip 10mm}
\newcommand{\bb}[1]{\mathbb{#1}}
\newcommand{\ra}{\rightarrow}
\newcommand{\wt}[1]{\widetilde{#1}}

\theoremstyle{plain}
\newtheorem{problem}{Problem}
\newtheorem{lemma}{Lemma}[problem]

\theoremstyle{remark}
\newtheorem{tpart}{}[problem]
\newtheorem*{ppart}{}

\begin{document}

\begin{problem}
  In the context of our proof of the zig-zag lemma.
  Prove that $\ker(\phi_\ast) \subset \mathrm{Im}(\partial_\ast)$ and $\ker(\psi_\ast) \subset \mathrm{Im}(\phi_\ast)$.
\end{problem}

\begin{proof}

\end{proof}

\begin{problem}
  Let $A: S^n \to S^n$ be the antipodal map. What is $A_\ast : H_n(S^n) \to H_n(S^n)$?
\end{problem}

\begin{proof}

\end{proof}

\begin{problem}
  Give a geometric description of the boundary map in the Mayer-Vietoris sequence.
\end{problem}

\begin{proof}

\end{proof}

\begin{problem}
  Using the Mayer-Vietoris sequence, compute the homology of the $n$-Sphere, $H_\ast (S^n)$.
\end{problem}

\begin{proof}
  We'll start by computing the reduced homology of $S^0$ and proceed by induction.
  The zeroth reduced homology group is the one less than the number of
  connected components copies of $\bb{Z}$. Since $S^0$ is two disjoint points we
  have that $\wt{H}_0(S^0)\cong\bb{Z}$.

  Next assume that $\wt{H}_{n-1}(S^{n-1})\cong\bb{Z}$ and is zero elsewhere.
  Then consider $S^n$ as the union of a point $*$ and $D^n$ enlarging them both slightly.
  Their intersection will be homotopy equivalent to $S^{n-1}$.
  Then using the Mayer-Vietoris sequence we have the long exact sequence
  \[
    \xymatrix{
      \cdots \ar[r] &H_n(S^{n-1}) \ar[r] & H_n(*)\oplus H_n(D^n)\ar[r] & H_n(S^n) \ar[r]
      &H_{n-1}(S^{n-1})\ar[r] & \cdots \\
    }
  \]
  All other portions of the sequence be either zero or $H_p(S^n)$ sandwiched between
  two zeros forcing it to be zero. Rewrite the above sequence with the portions
  we know and we get
  \[
    \xymatrix{
      0\ar[r] & H_n(S^n) \ar[r]& (H_{n-1}(S^{n-1})\cong\bb{Z})\ar[r] & 0\\
    }
  \]
  Which implies that $\wt{H}_n(S^n)\cong \wt{H}_{n-1}(S^{n-1})\cong\bb{Z}$.

  Therefore the homology of $S^n$ is
  \[ H_p(S^n)=\left\{
      \begin{array}{cc}
        \bb{Z}& p=n,0\\
        0 & \mathrm{else}\\
      \end{array}
    \right.\]
  except for $S^0$ which has $H_0(S^0)=\bb{Z}^2$ and 0 elsewhere.
\end{proof}

\begin{problem}
  Let $T^2 = S^1 \times S^1$ be the torus, and $h: S^1 \to T^2$ an embedding of the unit
  circle into $T^2$. Form the space 
  \begin{equation*}
    X = T^2 \cup_h D^2
  \end{equation*}
  by attaching a $2$-cell $D^2$ to $T^2$ via the map $h$. Compute the homology of $X$.
  Note that there is more than one case.
\end{problem}

\begin{proof}

\end{proof}


%%%%%%%%%%%%%%%%%%%%%%%%%%%%%%%%%%%%%%%%%%%%%%%%%%%%%%%%%%%%%%%%%%%%%%%%%%%%% 
\end{document}
