\documentclass[10pt]{article}
\usepackage[utf8]{inputenc}
\usepackage{amscd}
\usepackage{amsmath}
\usepackage{amssymb}
\usepackage{amsthm}
\usepackage{listings}
\usepackage{enumerate}
\usepackage[all,cmtip]{xy}

\textwidth=15cm \textheight=22cm \topmargin=0.5cm \oddsidemargin=0.5cm \evensidemargin=0.5cm

\newcommand{\sk}{\vskip 10mm}
\newcommand{\bb}[1]{\mathbb{#1}}
\newcommand{\ra}{\rightarrow}
\newcommand{\wt}[1]{\widetilde{#1}}

\theoremstyle{plain}
\newtheorem{problem}{Problem}
\newtheorem{lemma}{Lemma}[problem]

\theoremstyle{remark}
\newtheorem{tpart}{}[problem]
\newtheorem*{ppart}{}

\begin{document}

\begin{problem}
  In the context of our proof of the zig-zag lemma.
  Prove that $\ker(\phi_\ast) \subset \mathrm{Im}(\partial_\ast)$ and $\ker(\psi_\ast) \subset \mathrm{Im}(\phi_\ast)$.
\end{problem}

\begin{proof}
  The following diagram contains the necessary maps
  \[
    \xymatrix{
      & \vdots \ar[d]_\partial & \vdots \ar[d]_\partial & \vdots \ar[d]_\partial & \\
      0 \ar[r] & C_{p+1} \ar[r]^\phi \ar[d]_\partial & D_{p+1} \ar[r]^{\psi}\ar[d]_\partial
      & E_{p+1}  \ar[d]_\partial \ar[r] & 0\\
      0 \ar[r] & C_{p} \ar[r]^\phi \ar[d]_\partial & D_{p} \ar[r]^{\psi}\ar[d]_\partial
      & E_{p}  \ar[d]_\partial \ar[r] & 0\\
      0 \ar[r] & C_{p-1} \ar[r]^\phi \ar[d]_\partial & D_{p-1} \ar[r]^{\psi}\ar[d]_\partial
      & E_{p-1}  \ar[d]_\partial \ar[r] & 0\\
      & \vdots & \vdots & \vdots &
    }
  \]
  with the long exact sequence from the zig-zag lemma
  \[
    \xymatrix{
      \cdots \ar[r] & H_n(C) \ar[r]^{\phi_*} & H_n(D) \ar[r]^{\psi_*} & H_n(E) \ar[r]^{\partial_*}
      & H_{n-1}(C)\ar[r] & \cdots
    }
  \]

  First we'll show that $\ker(\phi_*)\subset\mathrm{Im}(\partial_*)$. Start with $[c]\in\ker(\phi_*)$.
  Then $\phi(c)=d$ where $[d]=0$. This implies that there is a $d'\in D_{p+1}$ such
  that $\partial d'=d$. In addition we have $\psi\circ\partial(d)=0$ and if we have $e:=\psi(d)$ then
  $\partial e =0$ by commutativity and thus $[e]$ is a class in homology. Let
  $e':=\psi(d')$. Then $\partial e' = e$ which implies that $[e']$ is a class in homology.
  Finally by commutativity we have that $\partial_*[e']=c$ which implies that
  $\ker(\phi_*)\subset\mathrm{Im}(\partial_*)$.

  Next we'll show that $\ker(\psi_*)\subset\mathrm{Im}(\phi_*)$. Let $[d]\in \ker(\psi_*)$. Then
  we know that $\partial d=0$ and that for $e:=\psi(d)$ that there exists $\partial e':= e$.
  Since $\psi$ is surjective we have $\psi(d'):=e'$. By commutativity we have
  that $\psi(d-\partial d')=0$ so there exists an $a$ such that $\phi(a)=d-\partial d'$ which is
  unique by injectivity of $\phi$. In addition $\partial a=0$ by commutativity.
  Then in homology we get that $\phi_*[a] = [d-\partial d']=[d]$. Therefore
  $\ker(\psi_*)\subset\mathrm{Im}(\phi_*)$.
\end{proof}

\begin{problem}
  Let $A: S^n \to S^n$ be the antipodal map. What is $A_\ast : H_n(S^n) \to H_n(S^n)$?
\end{problem}

\begin{proof}
  We'll start by showing that the map $r_i:S^n\rightarrow S^n$ defined by
  \[ r(x_1,\ldots,x_i,\ldots,x_{n+1})=(x_1,\ldots,-x_i,\ldots,x_{n+1})\]
  has degree $-1$. Start with $S^n$ and consider it as two $D^n$, called $p$ and $q$,
  joined at their boundaries with the boundary coinciding with $x_i$.
  The only non-zero reduced homology for $S^n$
  is $\wt{H}_n(S^n)\cong\bb{Z}$ that is generated by $p+q$. Then $r_{i*}(p+q)=-(p+q)$
  since it will swap $p,q$ and their orientations. Therefore the degree of $r_i$
  is $-1$.

  We can express the map $A:S^n\rightarrow S^n$ as the composition $A=r_1\circ\cdots\circ r_{n+1}$. Thus the
  degree of the map $A$ is $(-1)^{n+1}$.

  Therefore $A_*:H_n(S^n)\rightarrow H_n(S^n)$ is the identity map when $n$ is odd and the
  inverse map when $n$ is even.
\end{proof}

\begin{problem}
  Give a geometric description of the boundary map in the Mayer-Vietoris sequence.
\end{problem}

Begin with our space $X=U\cup V$ satisfying the conditions for the Mayer-Vietoris
sequence. Start with a cycle $[h]\in H_n(X)$. Then we can write $h=u+v$ where $u,v$
lie wholly in $U$ and $V$ respectively. Since $h$ represents a class in homology
we have $\partial x = 0$. However this implies that $\partial u+\partial v = 0$ giving us that
$\partial u = -\partial v$. It then follows that $\partial u,\partial v\in U\cap V$ and thus $\partial_*[h]=[u]$.

\begin{problem}
  Using the Mayer-Vietoris sequence, compute the homology of the $n$-Sphere, $H_\ast (S^n)$.
\end{problem}

\begin{proof}
  We'll start by computing the reduced homology of $S^0$ and proceed by induction.
  The zeroth reduced homology group is the one less than the number of
  connected components copies of $\bb{Z}$. Since $S^0$ is two disjoint points we
  have that $\wt{H}_0(S^0)\cong\bb{Z}$.

  Next assume that $\wt{H}_{n-1}(S^{n-1})\cong\bb{Z}$ and is zero elsewhere.
  Then consider $S^n$ as the union of a point $*$ and $D^n$ enlarging them both slightly.
  Their intersection will be homotopy equivalent to $S^{n-1}$.
  Then using the Mayer-Vietoris sequence we have the long exact sequence
  \[
    \xymatrix{
      \cdots \ar[r] &H_n(S^{n-1}) \ar[r] & H_n(*)\oplus H_n(D^n)\ar[r] & H_n(S^n) \ar[r]
      &H_{n-1}(S^{n-1})\ar[r] & \cdots \\
    }
  \]
  All other portions of the sequence be either zero or $H_p(S^n)$ sandwiched between
  two zeros forcing it to be zero. Rewrite the above sequence with the portions
  we know and we get
  \[
    \xymatrix{
      0\ar[r] & H_n(S^n) \ar[r]& (H_{n-1}(S^{n-1})\cong\bb{Z})\ar[r] & 0\\
    }
  \]
  Which implies that $\wt{H}_n(S^n)\cong \wt{H}_{n-1}(S^{n-1})\cong\bb{Z}$.

  Therefore the homology of $S^n$ is
  \[ H_p(S^n)=\left\{
      \begin{array}{cc}
        \bb{Z}& p=n,0\\
        0 & \mathrm{else}\\
      \end{array}
    \right.\]
  except for $S^0$ which has $H_0(S^0)=\bb{Z}^2$ and 0 otherwise.
\end{proof}

\begin{problem}
  Let $T^2 = S^1 \times S^1$ be the torus, and $h: S^1 \to T^2$ an embedding of the unit
  circle into $T^2$. Form the space 
  \begin{equation*}
    X = T^2 \cup_h D^2
  \end{equation*}
  by attaching a $2$-cell $D^2$ to $T^2$ via the map $h$. Compute the homology of $X$.
  Note that there is more than one case.
\end{problem}

\begin{proof}
  We begin by using the Mayer-Vietoris sequence to decompose $X$ into $T^2$ and $D^2$
  to get an exact sequence describing our problem. The only nonzero portions of
  the sequence are
  \[
    \xymatrix{
    0 \ar[r] & H_2(T^2)\oplus H_2(D^2) \ar[r] & H_2(X) \ar[r] & H_1(S^1) \ar[r] &
    H_1(T^2)\oplus H_1(D^2) \ar[r] & H_1(X) \ar[r] & 0\\
    }
  \]
  Then filling in for some of the groups we have the exact sequence
  \[
    \xymatrix{
      0 \ar[r] & \bb{Z} \ar[r] & H_2(X) \ar[r] & \bb{Z} \ar[r]
      &\bb{Z}\oplus\bb{Z}\ar[r] & H_1(X)\ar[r] & 0
    }
  \]
  There are two cases to consider for the map $h$. The first is where $h$
  is homotopic to one of the generators of $H_1(T)$ and the latter is when $h$
  is null.
\end{proof}


%%%%%%%%%%%%%%%%%%%%%%%%%%%%%%%%%%%%%%%%%%%%%%%%%%%%%%%%%%%%%%%%%%%%%%%%%%%%% 
\end{document}
