\documentclass[10pt]{article}
\usepackage[utf8]{inputenc}
\usepackage{amscd}
\usepackage{amsmath}
\usepackage{amssymb}
\usepackage{amsthm}
\usepackage{listings}
\usepackage{enumerate}
\usepackage[all,cmtip]{xy}

\textwidth=15cm \textheight=22cm \topmargin=0.5cm \oddsidemargin=0.5cm \evensidemargin=0.5cm

\newcommand{\sk}{\vskip 10mm}
\newcommand{\bb}[1]{\mathbb{#1}}
\newcommand{\ra}{\rightarrow}
\newcommand{\rk}{\mathrm{rk}}

\theoremstyle{plain}
\newtheorem{problem}{Problem}
\newtheorem{lemma}{Lemma}[problem]

\theoremstyle{remark}
\newtheorem{tpart}{}[problem]
\newtheorem*{ppart}{}

\begin{document}

\begin{problem}
  Let $X$ denote the surface of genus two with a single boundary component.
  Let $A$ denote the boundary of $X$. Compute the relative homology groups
  $H_p(X,A)$.
\end{problem}

\begin{proof}
  The surface of genus two with a single boundary component will be a $2$-holed
  torus with a disk removed where the boundary of the disk is $A$. From Hatcher
  we have that $H_p(X,A)\cong \widetilde{H}_p(X/A)$. However $X/A$ is homeomorphic
  to the $2$-holed torus. As such the relative homology will be
  \[
    H_p(X,A)=
    \left\{
      \begin{array}{cr}
        0&p=0\\
        \bb{Z}\oplus\bb{Z}\oplus\bb{Z}\oplus\bb{Z} & p=1\\
        \bb{Z}&p=2\\
      \end{array}
    \right.
  \]
\end{proof}

\sk

\begin{problem}
  Show that $S^1\times S^1$ and $S^1\vee S^1\vee S^2$ have isomorphic homology groups,
  but that their universal covering spaces do not.
\end{problem}

\begin{proof}
  The first space $S^1\times S^1$ is the torus. As such its homology groups are
  \[ H_p(T^2)=
    \left\{
      \begin{array}{lr}
        \bb{Z}& p=0,2\\
        \bb{Z}\oplus\bb{Z}&p=1\\
        0&\mathrm{otherwise}
      \end{array}
    \right.
  \]
  For the latter space we can use the fact that the connected sum corresponds to
  the direct sum of the reduced homology groups. Since for $S^n$ the only nonzero
  reduced homology group is $H_n(S^n)=\bb{Z}$ we have
  \[
    H_p(S^1\vee S^1\vee S^2)=
    \left\{
      \begin{array}{lr}
        \bb{Z}& p=0,2\\
        \bb{Z}\oplus\bb{Z}&p=1\\
        0&\mathrm{otherwise}
      \end{array}
    \right.
  \]
  which shows that their homology groups are isomorphic.

  The universal cover of $S^1\times S^1\cong T^2$ is the grid $\bb{Z}\times\bb{Z}$ wherein
  each point is connected to its neighbors by a unit interval. The universal
  cover of $S^1\vee S^1\vee S^2$ will be the Cayley graph of $\bb{Z}*\bb{Z}$
  where each point has an $S^2$ attached to it. The covering space of the
  former does not have any two cells which implies that $\widetilde{H}_2(T^2)=0$.
  However since there are no three cells and each two sphere is attached at
  only one point $\widetilde{H}_2(S^1\vee S^1\vee S^2)$ will be nontrivial.

  Therefore the homology of the universal covers of $S^1\times S^1$ and $S^1\vee S^1\vee S^2$
  are not isomorphic even though their homologies are.
\end{proof}

\sk

\begin{problem}
  Group the following spaces into homotopy equivalence classes. Justify
  your answers.
  \begin{itemize}
  \item[(a)] M\"obius band
  \item[(b)] The torus
  \item[(c)] $B^2\times S^1$
  \item[(d)] The torus minus one point
  \item[(e)] The torus minus two points
  \item[(f)] The Klein bottle minus one point
  \item[(g)] $\bb{R}^3$ minus the $z$-axis
  \item[(h)] $\bb{R}^3$ minus the unit circle in the $xy$-plane, $\{x^2+y^2=0,z=0\}$
  \item[(i)] The intersection of $(g)$ and $(h)$
  \item[(j)] $S^3$ minus two linked circles
  \item[(k)] $S^3$ minus two unlinked circles
  \end{itemize}
\end{problem}

There are many pictures attached showing important points as well as
explanation attached below. The equivalence classes are
\[
  \begin{array}{c|c|c|c|c|c}
    S^1&T^2&S^1\vee S^1&S^1\vee S^1 \vee S^1& S^1\vee (S^1\vee S^2)& S^1\vee S^2\\
    \hline
    (a)&(b)&(d)&(e)&(k)&(h)\\
    (c)&(i)&(f)&&&\\
    (g)&(j)&&&&\\
  \end{array}
\]

\begin{itemize}
\item[(a)] For the M\"obius band there is a circle running along the center with
  an interval attached. We can squish the interval down to a point to get a circle.
\item[(b)] The torus is a torus.
\item[(c)] Each point of the circle has a disk attached that we can squish down to a
  point.
\item[(d)] If we draw the torus as a square with sides identified and poke a hole
  in the middle we can deform the punctured disk to its boundary which is
  $S^1\vee S^1$.
\item[(e)] Similar to $(d)$ except when we puncture two holes we have to retract the
  part to a line between the holes.
\item[(f)] This is the same process as the torus in that when we puncture the
  disk we can pull it back to the boundary.
\item[(g)] We can flatten it to $\bb{R}^2\setminus\{0\}$ which is homotopy equivalent to a
  circle.
\item[(h)] This space is almost a torus however having the $z$-axis left interferes
  with that. As such we can pull the space down to get an $S^2$ except with
  the two points on the $z$-axis attached. Pull them apart leaving an interval
  connecting them and then move the two attachment points together to get
  $S^1\vee S^2$.
\item[(i)] If we take a look at a vertical slice we get an open half plane
  minus a point. This will be homotopy equivalent to a circle. Since there
  is one for each point of the circle this is a Torus.
\item[(j)] Being in $S^3$, as opposed to $\bb{R}^3$ lets us effectively take one
  of the circles and turn it into a line by putting it through ``the point at
  infinity'' of $\bb{R}^3$. If we do this with two linked circles we get a line
  going through a circle which is a space homeomorphic to $(h)$.
\item[(k)] We do the same process as above except we end up with a line not
  going through the circle. This ends up being as shown in the picture
  the wedge of $\bb{R}^3$ with a line taken out and an $\bb{R}^3$ with the circle
  taken out. Thus giving us $S^1\vee (S^1\vee S^2)$.
\end{itemize}

\sk

\begin{problem}
  Compute the homology of the $CW$ complex obtained from the cube
  $I\times I\times I$ by identifying opposite faces after a $1/4$ twist.
\end{problem}

\newpage
\ 
\vskip3in

\begin{proof}
  The picture above is a CW complex $X$. The boundary maps
  will be
  \begin{align*}
    \partial I&= E-E+F-F+G=0\\
    \partial E&= a+b+c+d\\
    \partial F&= a-b-c+d\\
    \partial G&= a+b-c-d\\
    \partial a&= v-u\\
    \partial b&= u-v\\
    \partial c&= v-u\\
    \partial d&= u-v\\
    \partial u&= 0\\
    \partial v&= 0\\
  \end{align*}

  We'll start with $H_3(X)$. Since there are no 4-cells we have that
  $\mathrm{Im}(\partial_4)=0$ and since $\partial I=0$ it must be that $H_3(X)\cong\bb{Z}$.

  Next for $H_2(X)$ the image of $\partial_3$ is zero. As such $H_2(X)\cong \ker \partial_2$.
  If we look at $\partial_2$ as a matrix it will be
  \[
    \left(
      \begin{array}{rrr}
        1&1&1\\
        1&-1&1\\
        1&-1&-1\\
        1&1&-1\\
      \end{array}
    \right)
  \]

  If we reduce this to Smith normal form we get
  \[
    \left(
      \begin{array}{rrr}
        1&0&0\\
        0&2&0\\
        0&0&2\\
        0&0&0\\
      \end{array}
    \right)
  \]

  for which the kernel is zero. Therefore $H_2(X)=0$.

  Next we will compute $H_1(X)$. Using the matrix above we have that
  $\mathrm{Im}\partial_2=\langle a,2b,2c\rangle$. Since $\partial_1 a=\partial_1 c=-\partial_1 b=-\partial_1 d$ the
  kernel of $\partial_1$ will linear combinations where the number of $a,c$s and
  the number of $b,d$s are equal. If we mod this out by the image of
  $\partial_2$ we have $a=0$, $d$ is left alone, $2b=0$, and $2c=0$. However
  since $d$ and $c$ have to agree with the $a$s and $b$s we are left
  with $\langle b,c|2b=0,2c=0\rangle$. This gives us that
  $H_1(X)\cong \bb{Z}_2\oplus\bb{Z}_2$.

  Finally $H_0(X)\cong \bb{Z}$ as there is one connected component.

  Thus the homology groups for this space are
  \[
    H_p(X) = 
    \left\{
      \begin{array}{lr}
        \bb{Z}& p=0,3\\
        \bb{Z}_2\oplus\bb{Z}_2 & p=1\\
        0 & p=2\\
      \end{array}
    \right.
  \] 
\end{proof}

\sk

\begin{problem}
  A map $f:S^n\rightarrow S^n$ satisfying $f(x)=f(-x)$ for all $x\in S^n$ is called and
  \textit{even map}. Show that an even map $S^n\rightarrow S^n$ must have even degree,
  and that the degree must in fact be zero when $n$ is even. When $n$ is odd,
  show that there exist even maps of any given even degree.
  [Hint: First show that if $f$ is even, then it necessarily factors as a
  composition $S^n\rightarrow \bb{R}P^n\rightarrow S^n$.]
\end{problem}

\begin{proof}
  Let $f$ be a map from $S^n$ to $S^n$ such that $f(x)=f(-x)$ for all $x\in S^n$.
  Since $\bb{R}P^n$ is a quotient space of $S^n$ where antipodal points are
  identified any even map from $S^n$ respects equivalence classes for the
  quotient space and as such it factors as
  \[
    \xymatrix{
      S^n \ar[d]^q \ar[r]^f & S^n\\
      \bb{R}P^n \ar[ru]_{\widetilde{f}}
    }
  \]

  First note that $\deg q =2$. This is because
  for any point $x\in\bb{R}P^n$ it will have two points mapping to it from $S^n$ and
  it will be the identity map giving us local degrees of 1 which add up to 2. Since
  any even map will factor in this way any even map must have even degree.
  
  When $n$ is even $H_n(\bb{R}P^n)\cong\bb{Z}$ and when $n$ is even $H_n(\bb{R}P^n)\cong 0$.
  If $n$ is odd and $f$ is even then $\deg f = \deg q\cdot\deg \widetilde{f}$. However
  since $H_n(\bb{R}P^n)\cong 0$ when $n$ is even then the degree of $f$ has to be zero
  as $\deg \widetilde{f}=0$ since it is mapping out of the trivial group.

  Now suppose that $n$ is odd. There is a quotient map
  $r:\bb{R}P^n\rightarrow (\bb{R}P^n/\bb{R}P^{n-1}\equiv S^n)$. This map will have degree $1$ and
  as such $q\circ r$ will be a map from $S^n\rightarrow S^n$ of degree two. From there if we
  take a map of degree $k$, $f_k:S^n\rightarrow S^n$ (Hatcher $2.32$). Then $f_k\circ q\circ r$ will be
  an even map of degree $2k$.
\end{proof}

\sk

\begin{problem}
  Let $\mathcal{C}=(C_i,\partial)$ be a chain complex over $\bb{R}$ with only
  finitely many $C_i\neq 0$. Show that the following methods for computing
  Euler characteristic yield the same answer:
  \[ \chi(\mathcal{C})=\sum_i(-1)^i\mathrm{rk}C_i\]
  and
  \[ \chi(\mathcal{C})=\sum_i(-1)^i\mathrm{rk}H_i(\mathcal{C})\]
  The same is true with $\bb{Z}$ coefficients, but requires a tiny bit
  more thought.
\end{problem}

\begin{proof}
  Note that if any of the $C_i$s are infinite dimensional then the sum
  is infinite and the equality will hold.
  
  Otherwise suppose that all of the $C_i$s are of finite dimension and start
  with the sum using homology
  \[\chi(\mathcal{C})= \sum_i (-1)^i(\rk H_i(\mathcal{C})) \]
  Since $H_i(\mathcal{C})=\ker\partial_i/\mathrm{Im}\partial_{i+1}$ the sum is equivalent to
  \[ \sum_i(-1)^i(\rk\ker\partial_i-\mathrm{Im}\partial_{i+1}) \]
  As there are only finitely many $C_i$ that are nonzero we can rearrange
  the terms as we desire to get
  \[ \sum_i(-1)^i(\rk\ker\partial_i+\mathrm{Im}\partial_i)\]
  by shifting all of the images up one term. By the rank nullity theorem we
  then have
  \[ \sum_i(-1)^i\rk C_i \]
  which completes the proof.
\end{proof}

%%%%%%%%%%%%%%%%%%%%%%%%%%%%%%%%%%%%%%%%%%%%%%%%%%%%%%%%%%%%%%%%%%%%%%%%%%%%%
\end{document}
