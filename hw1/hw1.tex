\documentclass[10pt]{article}
\usepackage[utf8]{inputenc}
\usepackage{amscd}
\usepackage{amsmath}
\usepackage{amssymb}
\usepackage{amsthm}
\usepackage{listings}
\usepackage{enumerate}

\textwidth=15cm \textheight=22cm \topmargin=0.5cm \oddsidemargin=0.5cm \evensidemargin=0.5cm

\newcommand{\sk}{\vskip 10mm}
\newcommand{\bb}[1]{\mathbb{#1}}
\newcommand{\ra}{\rightarrow}

\theoremstyle{plain}
\newtheorem{problem}{Problem}
\newtheorem{lemma}{Lemma}[problem]

\theoremstyle{remark}
\newtheorem{tpart}{}[problem]
\newtheorem*{ppart}{}

\begin{document}

\begin{problem} %1
  Construct several examples of homotopic and non-homotopic maps.
\end{problem}

See attached.

\sk

\begin{problem} %2
  Show that the relation of fixed-endpoint homotopy is an equivalence relation.
\end{problem}

\begin{proof}
  To show that the relation of fixed-endpoint homotopy ($\sim$)is an equivalence relation
  we will show that it is reflexive, symmetric, and transitive.

  Let $f:X\rightarrow Y$ be a continuous function. Then we can construct a fixed endpoint
  homotopy from $f$ to $f$ via $H(x,t)=f(x)$ showing that $\sim$ is reflexive.

  Next let $f\sim g$. Then there is a fixed endpoint homotopy $H(x,t)$ from $f$ to $g$.
  Define $H':X\times [0,1]\rightarrow Y$ via $H'(x,t)=H(x,1-t)$. Then $H'$ is a homotopy
  from $g$ to $f$ that fixes the endpoints since $H$ did as well demonstrating that
  $\sim$ is reflexive.

  Finally suppose that $f\sim g$ and $g\sim h$. Then there are fixed endpoint
  homotopies $F$ from $f$ to $g$ and $G$ from $g$ to $h$. Define a new homotopy
  via
  \[
    H(x,t) = \left\{
      \begin{array}{cc}
        F(x,2t) & 0\leq t\leq \frac{1}{2}\\
        G(x,2t-1) & \frac{1}{2}\leq t\leq 1\\
      \end{array}
      \right.
    \]
    This homotopy will fix the endpoints since the two it is constructed from do.

    Therefore since $\sim$ is reflexive, symmetric, and transitive it is an
    equivalence relation.
\end{proof}

\sk

\begin{problem} %3
  Construct some examples of paths which are fixed-endpoint homotopic, and some
  which are not.
\end{problem}

See attached.

\sk

\begin{problem} %4
  \begin{itemize}
  \item[a)] Show that any convex open subset of $\bb{R}^n$ is contractible.
  \item[b)] Show that a contractible space is path connected.
  \item[c)] Show that if $Y$ is contractible, then all maps
    \[ f:X\rightarrow Y\]
    are homotopic.
  \item[d)] Show that if $X$ is contractible and $Y$ is path-connected, then all
    maps
    \[ f:X\rightarrow Y\] are homotopic. What happens if we remove the path-connectedness
    assumption?
  \end{itemize}
\end{problem}

\begin{proof}
  \begin{itemize}
  \item[a)] Let $U$ be a convex open subset of $\bb{R}^n$ and let $u\in U$. 
    Given any point $x$ we can construct a path $p_x$ from $x$ to $u$ via the definition
    of convexity. Then construct a homotopy from $id_X$  to the constant map at $u$
    ($c_u$) via $H(x,t)=p_x(t)$.

    Therefore since $id_X\sim c_u$ for $U$, any open convex subset of $\bb{R}^n$ is
    contractible.
  \item[b)] Let $X$ be a contractible space and let $x,y\in X$ be points in $X$.
    Since $X$ is contractible we have a homotopy $H$ from $id_X$ to $c_z$ for some
    point $z\in X$. Define a path from $x$ to $z$ via $p_x(t)=H(x,t)$ and a path
    from $z$ to $y$ via $p_y(t)=H(y,(1-t))$. Then the path $p_x*p_y$ is a path
    from $x$ to $y$.

    Therefore since a path exists between any two point in a contractible space $X$,
    $X$ is path connected.
  \item[c)] Let $Y$ be a contractible space and $f,g:X\rightarrow Y$. Since $Y$ is
    contractible there exists a homotopy $H$ from $id_Y$ to a constant function
    $c_y$. Define $H_f(x,t)=H(f(x),t)$ and $H_g(x,t)=H(g(x),t)$. Then define
    a homotopy $F:X\times[0,1]\rightarrow Y$ via
    \[
      F(x,t) = \left\{
        \begin{array}{cc}
          H_f(x,2t) & 0\leq t \leq \frac{1}{2}\\
          H_g(x,2(1-t)) & \frac{1}{2}\leq  t\leq 1\\
        \end{array}
        \right.
      \]

      Therefore any two maps into a contractible space are homotopic.
    \item[d)] Let $X$ be a contractible space and $Y$ a path-connected space. Since
      $X$ is contractible there is a homotopy $H$ from $id_X$ to a constant map $c_{x_0}$.
      Let $f,g:X\rightarrow Y$ and $p$ a path from $f(x_0)$ to $g(x_0)$. Then define
      a homotopy $F:X\times[0,1]\rightarrow Y$ from $f$ to $g$ via
      \[
        F(x,t) = \left\{
          \begin{array}{cc}
            f(H(x,3t)) & 0\leq t\leq \frac{1}{3}\\
            p(3t-1) & \frac{1}{3}\leq t\leq\frac{2}{3}\\
            g(H(x,3-3t)) & \frac{2}{3}\leq t \leq 1\\
          \end{array}
        \right.
      \]

      Therefore if $X$ is contractible and $Y$ is path connected any two maps
      $f:X\rightarrow Y$ are homotopic.

      This does not hold if we remove the path connected assumption. Let
      $\bb{R}_2=\bb{R}\times\{0\}\cup \bb{R}\times\{1\}$. Then
      the map $f(x)=(x,0)$ is not homotopic to $g(x)=(x,1)$ since there is no path
      between the two copies of $\bb{R}$.
  \end{itemize}
\end{proof}

\sk

\begin{problem} %5
  Check that the fundamental group of a pointed space $(X,x_0)$ is a group.
\end{problem}

\begin{proof} To show that $\pi(X,x_0)$ is a group we need to verify that
  that path concatenation is associative, there is an identity, and that there are inverses.
  We denote the equivalence class of a loop $\alpha$ via $[\alpha]$
  \begin{itemize}
  \item Let $[\alpha],[\beta],[\gamma]\in\pi(X,x_0)$. Then we can define a fixed endpoint
    homotopy between $(\alpha*\beta)*\gamma$ and $\alpha*(\beta*\gamma)$ via
    \[
      H(t,s) = \left\{
        \begin{array}{cc}
          \alpha(\frac{4t}{s+1}) & 0\leq t \leq \frac{s+1}{4}\\
          \beta(4t-s-1) & \frac{s+1}{4}\leq t\leq \frac{s+2}{4}\\
          \gamma(\frac{4t-s-2}{2-s}) & \frac{s+2}{4} \leq t \leq 1
        \end{array}
      \right. 
    \]
    Since there is a fixed endpoint homotopy between them they lie in the same
    equivalence class of loops.

    Therefore the concatenation operation on loops is associative.
  \item Let $[\alpha]\in\pi(X,x_0)$ and let $p_{x_0}$ be the constant loop at $x_0$.
    Consider the loop $\alpha*p_{x_0}$. We can then construct a fixed endpoint homotopy from
    $\alpha*p_{x_0}$ to $\alpha$ via \textbf{I think the second line is wrong}
    \[
      H(t,s)=\left\{
        \begin{array}{cc}
          \alpha(\frac{2t}{1+s}) & 0 \leq t \leq \frac{1+s}{2}\\
          p_{x_0}(\frac{2t}{1+s}-1) & \frac{1+s}{2}\leq t\leq 1\\
        \end{array}
      \right.
    \]

    Since there is a fixed endpoint homotopy between them they lie in the same equivalence
    class of loops and the constant loop is a right identity.
    
    Therefore there exists a right identity in $\pi(X,x_0)$.
  \item Let $[\alpha]\in\pi(X,x_0)$. Then define $\beta(t)=\alpha(1-t)$. We can define
    a homotopy from $\alpha*\beta$ to $p_{x_0}$ via
    \[
      H(t,s) = \left\{
        \begin{array}{cc}
          \alpha(2(1-s)t) & 0 \leq t \leq \frac{1}{2}\\
          \beta(2(1-s)t+(2s-1)) & \frac{1}{2}\leq t\leq 1\\                            
        \end{array}
      \right.
    \]

    Since there is a fixed endpoint homotopy between them they lie in the same equivalence
    class of loops which implies that $\beta$ is a right inverse of $\alpha$.

    Therefore given $[\alpha]\in\pi(X,x_0)$ there exists a right inverse.
  \end{itemize}

  Therefore since loop concatenation is associative, there exists an identity element,
  and any class of loops has an inverse we have that $\pi(X,x_0)$ is a group under the
  loop concatenation operation.
\end{proof}

\sk

\begin{problem} %6
  Show that if $x_0,x_1$ are in the same path component of a space $X$, then
  $\pi_1(X,x_0)\simeq\pi_1(X,x_1)$.
\end{problem}

\begin{proof}
  Let $p$ denote a path from $x_1$ to $x_0$ and let $p_r=p(1-t)$. Then define
  a map $\phi([\alpha]):\pi(X,x_0)\rightarrow\pi(X,x_1)$ via
  \[ \phi([\alpha])=[p^r*\alpha*p]\]
  We can define an inverse to $\phi$ as
  \[ \phi^{-1}([\alpha])=[p*\alpha*p^r]\]
  To verify for that it is indeed an inverse we can check
  \[ \phi\circ\phi^{-1}([\alpha])=[p^r*p*\alpha*p^r*p]=[\alpha]\]
  and
  \[ \phi^{-1}\circ\phi([\alpha])=[p*p^r*\alpha*p*p^r]=[\alpha]\].
  Therefore $\phi$ is a bijection.

  To show that $\phi$ is a homomorphism we will show that $\phi([\alpha])*\phi([\beta])=
  \phi([\alpha]*[\beta])$ via
  \[ \phi([\alpha])*\phi([\beta])=[p^r*\alpha*p]*[p^r\beta*p] =
    [p^r*\alpha*p*p^r*\beta*p]=[p^r\alpha*\beta*p]=\phi([\alpha*\beta])=
    \phi([\alpha]*[\beta])\]
  Since $\phi$ preserves the group operation it is indeed a homomorphism and thus
  and isomorphism.

  Therefore if $x_0,x_1$ are in the same path component of a space $X$ then
  $\pi_1(X,x_0)\simeq\pi_1(X,x_1)$.
\end{proof}

\sk
%%%%%%%%%%%%%%%%%%%%%%%%%%%%%%%%%%%%%%%%%%%%%%%%%%%%%%%%%%%%%%%%%%%%%%%%%%%%%
\end{document}
